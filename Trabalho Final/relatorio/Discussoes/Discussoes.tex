\section{Discussões}

\subsection*{Experimento 1}

Nesse experimento podemos ver visualmente que o efeito da regularização não afetou muito os resultados de classificação obtidos, o que é um resultado desejado
visto que conseguimos reduzir a complexidade do modelo e ainda obter a resposta dentro do esperado.

\subsection*{Experimento 2}

Nesse experimento o foco foi na curva de soluções pareto-ótimas do problema. Qualquer solução escolhida nessa curva é um ponto ótimo entre o tradeoff de erro de predição e complexidade do modelo.
Soluções acima dessa curva são consideradas soluções não otimizadas e as soluções abaixo da curva são não tangíveis ao problema e modelo utilizados. (considerando a quantidade de clusters da rede RBF como característica do modelo escolhido).

\subsection*{Experimento 3 e 4}

Nos experimentos utilizando o dataset do breast cancer os resultados com relação a regularização não foram conforme o esperado. No experimento 3 a acurácia subiu conforme aumentou o valor de lambida. Esse resultado é o oposto do esperado visto que estamos conseguindo melhorar os dois objetivos, portanto ainda não
chegamos na região ótima de soluções. Já no experimento 4 podemos ver que a curva não tem o formato desejado igual a vista no experimento 2. Dessa forma, podemos deduzir que o modelo e a quantidade de clusters selecionada não conseguiu aproximar de um conjunto de soluções pareto-ótimas.
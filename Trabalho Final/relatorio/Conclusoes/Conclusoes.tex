\section{Conclusões}

O Trabalho tem como objetivo fazer um estudo, compreensão e testes de regularização de redes neurais artificiais.
Durante o desenvolvimento do trabalho foram realizadas várias pesquisas sobre o funcionamento e implementação desse método. 
Fiz uma busca de pacotes de modelos prontos de ELM e RBF com regularização em python mas nenhuma chegou a gerar um experimento interessante. Muitos dos exemplos encontrados 
eram realizados utilizando outros métodos de reconhecimento de padrões, como SVM e Regressão Linear para estudar efeitos de regularização. Por esse motivo, o trabalho foi desenvolvido
implementando a metodologia de weight decay explicitada no Livro texto\cite*[]{LivroTexto} na minha implementação do algoritmo de RBF na linguagem python.

Conseguimos chegar a resultados esperados em algumas bases de dados e em resultados não desejados no dataset Breast Cancer. Pude fazer
uma avalição do porque o resultado divergiu. 

Por fim, com esse trabalho temos uma maior compreensão sobre o tema de regularização e otimização de modelos de redes neurais artificiais.

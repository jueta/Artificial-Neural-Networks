\section{Revisão de trabalhos correlados}

O algoritmo de regularização é encontrado em diversas áreas de estudo, porém com nomes diferentes. Na estatística é conhecido como \emph{Ridge regression}, em otimização de sistemas é \emph{Regularização de Tikhonov} e em aprendizado de máquina como \emph{weight decay}.
O conceito de regularização abrange também outros algoritmos de otimização. Dentre eles o \emph{Pruning}\footcite{Pruning (poda) algoritmo muito utilizado em Arvore de decisao}, OBD\footcite{OBD: Optimal Brain Damage}, OBS\footcite{OBS: Optimal Brain Surgeon} dentro outros.

Na tese de pós-graduação do autor Medeiros\cite*{OBDCAPES} há a proposta de um novo método de poda no qual ele nomeou de CAPE. Esse método é implementado em redes MLP\footcite{MLP: Multilayer Percepetron}
e tem por objetivo reduzir o custo computacional da inversão da matriz Hessiana, cálculo esse muito utilizado em outros algoritmos. Em seguida o autor implementa o algoritmo em um caso real de identificação de falhas em motores de indução para avaliar e comparar a performance em relação ao algoritmo OBS.







\section{Artigos Estudados}

O artigo \emph{Neural Networks Regularization With Graph-Based Local Resampling}\cite{AlexAssis}
se refere ao tratamento dos dados utilizando o Grafo de Gabriel\cite{GabrielGraph1}
antes do treinamento e inferência de redes neurais baseadas em perceptron com projeções randômicas (ELM).
Método esse chamado de \emph{Graph-based Local Resampling of perceptron-like neural networks with random projections (RN-ELM)}.
O algoritmo consiste em tratar os dados geometricamente para o treinamento da rede neural. Problemas de classificação multi-objetivo\footcite{Problemas multiobjetivo: Problemas com mais de um objetivo},
quando aplicado esse método, possuem boa eficiência devido a complexidade da rede treinada ser reduzida quando os dados de entrada são tratados.

O artigo \emph{Large Margin Gaussian Mixture Classifier With a Gabriel Graph Geometric Representation of Data Set Structure}\cite{LuizBambirra}
explicita o uso de Grafo de Gabriel para modelagem dos padrões de entrada nos problemas de classificação. O modelo é capaz de solucionar problemas
de redes neurais que tem por objetivos encontrar uma solução ótima para maior capacidade de generalização e menor complexidade do algoritmo. Sendo assim problemas classificados como
multiobjetivos. Problemas assim costumam ter não só uma solução ótima, mas um conjunto de soluções ótimas, denominado conjunto Pareto-Ótimas (PO).
O algoritmo consiste nas etapas de construção geométrica do grafo sobre as amostras, remoção dos ruídos e, em seguida, detecção de bordas. Por fim, define-se
a região de separação entre os conjuntos. 

\begin{figure}[h!]
    \centering
    \includegraphics[scale=0.5]{images/GabrielGraph.png}
    \caption{Exemplo de Grafo de Gabriel para o dataset Two Moons.}
\end{figure}

    O artigo \emph{Enhancing Performance of Gabriel Graph-Based Classifiers by a Hardware Co-Processor for Embedded System Applications}\cite{HardwareGabrielGraph}
tem uma abordagem similar ao objetivo desse trabalho. Nele foi analisada a performance de um classificador utilizando Grafo de Gabriel implementado em Hardware Embarcado.
A aplicação foi implementada em um FPGA\footcite{FPGA: Field-programmable gate array}, onde se é capaz de reorganizar blocos lógicos de processamento de forma a obter um Hardware dedicado 
para aquela tarefa. Como a implementação de Redes Neurais requer elevado cálculo matricial e paralelo, a implementação em Hardware dedicado para tal modelo apresenta boa performance quando se
comparado à implementação em software de alto nível. A linguagem de descrição de Hardware utilizada para programação do FPGA foi VHDL.\footcite{VHDL: VHSIC Hardware Description Language.}

Esse artigo se difere com projeto desejado no que se diz ao Hardware escolhido. No artigo foi-se utilizado um hardware capaz de processar a rede de forma mais performática. No projeto proposto
nesse trabalho o hardware utilizado será de Arquitetura ARM\footcite{ARM is a family of reduced instruction set computer (RISC) instruction set architectures for computer processors} com baixa capacidade de processamento.
Por esse motivo, o objetivo não é ter um resultado melhor do que os implementados em computadores pessoais mas sim em obter um resultado aceitável para a utilização em uma aplicação distribuída de baixo custo.

Por fim, foi estudado diversos projetos utilizando o microcontrolador ESP32 e algoritmos de predição utilizando Redes Neurais Artificiais. Portanto, creio ser um projeto realizável.

